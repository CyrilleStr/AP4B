\newpage
\section{Règle du jeu}

Le but du jeu est de garder le bonheur (« happiness ») des habitants au-dessus de 0 le plus longtemps possible. Pour cela , les habitations doivent être suffisamment alimentées en énergie et non polluées. Le joueur perd quand le bonheur arrive à zéro, son score équivaut au nombre de jours qu’ils a pu tenir.

Tous les bâtiments doivent obligatoirement être accessibles et donc placés à côté d’une route, les nouvelles routes placées doivent également suivre des routes déjà existantes.

\textbf{Production d’Energie :} 
	Il existe 2 types de bâtiments produisant de l’énergie : les « fossil plants » (centrale à gaz, charbon) et les renouvelables (solaire, éolien, hydraulique). Les « fossil plants » nécessitent d’être alimentés en ressources qui sont dispersées autour de la carte et peuvent être extraites grâce à des mines. L’activité de ces « fossil plants » induit de la pollution sur les 9 cases autour de celle-ci.
	
\textbf{Consommation d’énergie :}
	Deux types de bâtiments consomment également de l’énergie : les mines et les maisons. Une fois l’énergie de la journée produite, elle est d’abord distribuée aux mines en fonction de leurs besoin et l’énergie qui reste est ensuite divisée entre les maisons. Le bonheur est basé sur l’énergie que ces maisons reçoivent.
	
\textbf{Gestion de l’argent :}
	Au début du jeu, le joueur commence avec une somme d’argent qui lui permet de commencer à placer des bâtiments. Pour continuer à en gagner, il doit garder les habitants heureux car le bonheur est rapport de l’argent gagné par les maisons chaque jour. En plus du coût de placement des bâtiments, les centrales et les mines ont un « servicing cost » ou coût d’entretien qui consomme de l’argent chaque jour.
	
\textbf{Bâtiments inactifs :}
	Un bâtiment peut devenir inactif de 2 manières soit il n’a plus assez d’énergie pour fonctionner, soit son coût d’entretien est supérieur à l’argent disponible. Une fois inactif il ne produit plus de ressources ni d’énergie.
	
\textbf{Gestion du bonheur :}
	Le bonheur des habitant est basé sur le niveau d’alimentation en énergie des maisons. Exemple : Si la quantité d’énergie requise est atteinte à 100\% le bonheur de la maison est à 100\%, si elle n’est atteinte qu’à 30\%, la maison aura un niveau de bonheur de 30\%. Le bonheur global est la moyenne du bonheur de toutes les maisons. Si une maison est dans un nuage de pollution d’une « fossil plant », son bonheur baisse automatiquement de 5 points.
