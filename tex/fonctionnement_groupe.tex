\newpage
\section{Fonctionnement du groupe}

Notre groupe étant composé de quatre personnes, ils nous a paru crucial d'avoir une organisation bien claire et bien structuré en tout point. Nous nous sommes inspiré du fonctionnement des entreprises pour s'organiser. Voici quelque explications.

\begin{itemize}
    \item[\textbullet]
    \textbf{Outils utilisés} : 
    Nous avons créé un serveur Discord pour communiquer facilement et structurer nos échanges lors d'appel ou de messages. Nous avons choisi l'IDE \textit{Intellij}, dans la mesure où il est disponible gratuitement pour les étudiants et facile d'utilisation, il comprend beaucoup de fonctionnalités intéressantes tel que CodeWithMe, permettant de programmer à plusieurs en temps réel, la génération automatique des diagrammes à partir des classes Java et inversement pour une meilleure vue d'ensemble du projet et un bon gestionnaire de Git. Pour la bibliothèque graphique, nous avons opté pour \textit{JavaFX} car cette bibliothèque est nativement supportée par Intellij et permet l'édition des scènes très simplement via le logicel \textit{SceneBuilder} développé pour cela. Il permet de générer des vues en fichier \enquote{.fxml} qui sont ensuite utilisés par JavaFX pour afficher du contenu.
    \linebreak
    
    \item[\textbullet]
    \textbf{Protocoles} : 
    Pour éviter au maximum les conflits et pour garder en permanence une version exploitable de notre programme nous avons décidé d'attribuer une branche de développement \textit{git} par personne et de ne jamais ajouter du code directement sur la branche \textit{main}. À chaque nouvelle ajout de fonctionnalité, nous devions le faire sur notre branche respective et faire une \textit{pull request} pour signaler cet ajout. Puis au moins un autre membre devait vérifier les changements et valider cette \textit{pull request}. Ainsi, le programme de la branche \textit{main} possède une version stable et fonctionnelle.
    \linebreak
    
    \item[\textbullet]
    \textbf{Planification} :
    Dès le commencement de l'UV \textit{AP4B} nous avons planifié des réunions hebdomadaires le mardi matin. Ces réunions nous permettaient de prendre des décisions sur la conception, de poser des questions lors des Travaux Pratiques et de se répartir des tâches jusqu'à la prochaine réunion. Évidemment à l'approche de la date limite du rendu, des réunions supplémentaires se sont imposées afin de convenir et de respecter les délais.
    \linebreak
    
    \item[\textbullet]
    \textbf{Étapes} : nous avons réaliser ce projet en plusieurs étapes. Tous d'abord il a fallu transcrire toutes nos idées par écrit et faire un tris en fonction de la faisabilité de chacune. Puis, il a fallut dessiner nos diagrammes en commençant par les Use-Case. Après avoir rédiger un premier diagramme des classes nous avons commencé à coder. La priorité était d'implémenter la génération d'une carte et le temps. Puis nous avons ajouté des bâtiments et la gestions des ressources. Tout au long de la réalisation nous avons constamment remis en question le diagramme des classes et les use-cases. Ayant atteint nos objectifs avant les vacances de Noël, nous avons prévu d'implémenter des fonctionnalités \enquote{bonus}, mais utiles, pendant les vacances.
\end{itemize}